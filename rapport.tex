\documentclass{report}

\title{LU2IN006 Rapport du projet}
\author{Oleksandr Hoviadin, Tianxiang Xia}
\date{\today}

\begin{document}
\maketitle
\section{Sujet synthétique}
On fait un système de gestion des versions des fichiers
qui possède les fonctionnalités principales de git. sauvegarde des instantanés
des fichiers, navigation entre différentes versions
création et gestion des branches afin de trvailler simultanément
sur plusieures versions des projets et reunir les branches
en une version finale.

\section{Sur l'implémentation}
\begin{itemize}
  \item Si on suit l'énoncé, si on fait des add, après on fait une modification des fichiers,
  puis on fait un list-add, les informations sont vielles par rapport à la modification, or quand on
  fait commit, c'est la version après la modification qui sera soumise. Pour cette raison, on écrit les
  fonctions \verb|wfts_upd| et \verb|wtts_upd| qui permettent d'avoir des informations actualisées sur
  les fichiers.
  \item On stocke tous les nombres concernants chmod en octet, e.g. \verb|"%o"|.
  \item On stocke tous les fichiers auxilliaires du programme dans le dossier .myGit
  du répertoire utilisateur, une fois pour toute, on l'ignore dans \verb|listdir|.
  \item L'utilisation des macros dans \verb|misc.h| rend le programme très configurable.
  \item On fait un maximal d'abstraction que possible: on réutilise des fonctions existantes et
  on fait des petits outils en plus de l'énoncé e.g. \verb|s2f| et \verb|f2s|.
  \item On écrit des commentaires brèves pour les fonctions.
  \item On fait attention aux ownerships de la mémoire: chaque fonction, si elle est donnée de la mémoire pointée
  par un pointeur, cette mémoire doit être librérée ou donnée à une autre fonction avant qu'elle termine de sorte
  que quand le programme termine, on puisse garantir que toute la mémoire allouée est libérée. En pratique, on utilise
  \verb|const| pour indiquer le donné de ownership, dans des cas spécifiques, on utilise des commentaires ou il est
  clair par l'usage de la fonction.
  \item On a ajouté command \verb|myGit log| pour pouvoir consulter l'information du dernier commit (ainsi on
  peut utiliser l'entré \verb|second_predecessor|, qui est demandé d'être stocké par l'énoncé mais non utilisé).
\end{itemize}

\section{Structure de développement}

\begin{verbatim}
bin/ # les fichiers compilés
src/ # de la sources code
tests/ # les tests
Makefile
rapport.tex
\end{verbatim}

\section{Structure des fichiers générés par le programme}

\section{Mode de test et démonstration d'utilisation}

\paragraph{Test des unités} \verb|tests/tests.c|
\paragraph{Test rassemblé/démonstration d'utilisation} \verb|tests/assembled_tests?.sh|, lancer
les scripts dans \verb|tests/freespace|

Lancer sans paramètre pour consulter tous les commandes possibles.

\end{document}