\documentclass{report}

\title{LU2IN006 Rapport du projet}
\author{Oleksandr Hoviadin, Tianxiang Xia}
\date{\today}

\begin{document}
\maketitle
\section*{Sujet synthétique}
On crée un outil de gestion des versions des fichiers
qui possède les fonctionnalités principales de git : sauvegarde des instantanés
des fichiers, navigation entre différentes versions,
création et gestion des branches afin de travailler simultanément
sur plusieures versions des projets et avoir la possibilité de réunir les branches en une version finale.

\section*{Structure de développement}

\begin{verbatim}
bin/ # fichiers compilés/exécutables
src/ # fichiers source
tests/ # jeu de tests
tests/freespace/ # vide réservé pour tester
Makefile
\end{verbatim}

\section*{Structure de code}
\begin{itemize}
    \item \textbf{hashFunc.c} : contient les fonctionalités pour créer et manipuler les hashes.
    \item \textbf{cellList.c} : contient les fonctionalités pour manipuler des listes des chaînes des caractères : \verb|Cell|, \verb|List| (liste simplement chaînée), etc.
    \item \textbf{fsop.c} : contient les fonctionalités pour manipuler des fichiers et des dossiers : \verb|listdir|, \verb|blobFile|, etc.
    \item \textbf{workTree.c} : contient les fonctionalités pour manipuler des ensembles de fichiers : \verb|WorkFile|,
    \verb|WorkTree| (arbre des dossiers et des fichiers comme feuilles),
    \verb|saveWorkTree|, \verb|restoreWorkTree|, etc.
    \item \textbf{gestionCommits.c} : contient les fonctionalités permettant de naviguer entre des instantanés, de construire
    et de maintenir des graphes toplogiques des différentes versions de projet :
    \verb|Commit| (table de hachage), etc.
    \item\textbf{myGit.c} : sur l'interface utilisateur et les interactions : \verb|main|, etc.
    \item\textbf{misc.c} : les fonctions/macros supplémentaires qui sont pourtant très utiles.
\end{itemize}

\section*{Sur l'implémentation}
\begin{itemize}
  \item Selon l'énoncé, après avoir ajouté les fichiers dans la zone de préparation on peut encore les modifier et quand on
  fait commit c'est la version la plus récente qui sera soumise. Pour cette raison, on a défini les
  fonctions \verb|wfts_upd| et \verb|wtts_upd| qui permettent de retourner l'état actuel des fichiers en chaîne de caractères.
  \item On stocke tous les nombres concernants mode en octal, e.g. \verb|"%o"|.
  \item On stocke tous les fichiers auxilliaires du programme dans le dossier .mygit
  du répertoire utilisateur. Une fois pour tout, on l'ignore dans \verb|listdir|.
  \item L'utilisation des macros dans \verb|misc.h| rend le programme très configurable.
  \item On fait un maximum d'abstraction que possible: on réutilise des fonctions existantes et
  on fait des petits outils en dehors de l'énoncé qui peuvent beaucoup aider e.g. \verb|s2f|, \verb|f2s|, \verb|blobStringExt| (consulter les commentaires pour savoir l'usage des fonctions).
  \item On écrit des commentaires brèves pour les fonctions.
  \item On fait attention aux ownerships de la mémoire: chaque fonction, si elle est donnée de la mémoire pointée
  par un pointeur, cette mémoire doit être librérée ou donnée à une autre fonction avant qu'elle termine de sorte
  que quand le programme termine, on puisse garantir que toute la mémoire allouée est libérée. En pratique, on utilise
  \verb|const| pour indiquer l'information sur ownership, dans des cas spécifiques, on utilise des commentaires ou il est
  clair par l'usage de la fonction.
  \item On a ajouté command \verb|myGit log| pour pouvoir consulter l'information du dernier commit (ainsi on
  peut utiliser l'entrée \verb|second_predecessor|, qui est demandé d'être stockée par l'énoncé mais n'est pas utilisé).
  \item Dans cette application, l'éfficacité n'est pas la plus importante, on fait quand même beaucoup d'optimisations sans changer que
  l'application soit robust : on utilise \verb|buf[MAXL]| où la taille de mémoire est moins importante (la pile est plus rapide que le tas) vice versa; 
  la complexité temporelle est toujours optimale sous les structures données, etc.
\end{itemize}

\section*{Mode de test et démonstration d'utilisation}

\paragraph{} Créer le dossier \verb|bin| et lancer \verb|make| ou \verb|make release|
pour avoir \verb|myGit.exe| (version de développement) ou \verb|myGit| (version finale) dans \verb|bin|.

\paragraph{} Pour consulter toutes les commandes possibles, veuillez lancer \verb|myGit| sans paramètre.

\paragraph{Test des unités} \verb|tests/tests.c|
\paragraph{Test rassemblé/démonstration d'utilisation}\verb|tests/assembled_tests*.sh|, 
on lance les scripts dans \verb|tests/freespace| (créez-le s'il n'existe pas déjà)

\end{document}
