\documentclass{report}

\title{LU2IN006 Rapport du projet}
\author{Oleksandr Hoviadin, Tianxiang Xia}
\date{\today}

\begin{document}
\maketitle
\section{Sujet synthétique}
On fait un système de gestion des versions des fichiers
qui possède les fonctionnalités principales de git. sauvegarde des instantanés
des fichiers, navigation entre différentes versions
création et gestion des branches afin de trvailler simultanément
sur plusieures versions des projets et reunir les branches
en une version finale.

Taper \verb|myGit| pour consulter tous les commandes possibles.
\section{Sur l'implémentation}
\begin{itemize}
  \item Si on suit l'énoncé, si on fait des add, après on fait une modification des fichiers,
  puis on fait un list-add, les informations sont vielles par rapport à la modification, or quand on
  fait commit, c'est la version après la modification qui sera soumise. Pour cette raison, on écrit les
  fonctions \verb|wfts_upd| et \verb|wtts_upd| qui permettent d'avoir des informations actualisées sur
  les fichiers.
  \item On stocke tous les nombres concernants chmod en octet, e.g. \verb|"%o"|.
  \item On stocke tous les fichiers auxilliaires du programme dans le dossier .myGit
  du répertoire utilisateur, une fois pour toute, on l'ignore dans \verb|listdir|.
\end{itemize}

\section{Structure de développement}

\end{document}